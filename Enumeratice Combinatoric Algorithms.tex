%%!TEX TS-program = latex

\documentclass[11pt]{article}
\usepackage[a4paper]{geometry}
%\usepackage[pdftex]{graphicx}
\usepackage[utf8]{inputenc}
\usepackage{palatino}

\usepackage{amssymb}
\usepackage{amsmath}
\usepackage{amsthm}

\usepackage{caption}
\usepackage{subcaption}

\usepackage{array}

\usepackage{longtable}

% Graphen etc.
\usepackage{pst-node}
\usepackage{pst-tree}
\usepackage{pst-poly}

\newcommand{\rcirclenode}[3]{
	\circlenode{\# 1}{\parbox{\# 2}{\centering\# 3}}
}

\usepackage{hyperref}

\begin{document}

\setlength{\parindent}{0pt}


\title{Enumerative Combinatoric Algorithms}
\author{}
\date {}

\maketitle

\tableofcontents

\newpage

\section{Enumerating vs. Counting}


\subsection{Example 1: Spanning Paths}

\paragraph{Given:} $ n $ points in convex position

\paragraph{Question:} How many straight line planar (crossing-free) spanning paths (all points are used) are there? \newline

\begin{itemize}

\item $ n = 1 $

$ \square $

\item $ n = 2 $ \newline

\psmatrix[mnode=circle]
$ \cdot $ & $ \cdot $
\psset{shortput=nab,arrows=-}
\small
\ncline[linestyle=dotted,arrows=-,nodesep=15pt]{1,5}{1,7}
\ncline{1,1}{1,2}
\endpsmatrix \newline

\item $ n = 3 $: 4 paths

\psmatrix[mnode=circle,colsep=0.2,rowsep=0.2]
& $ \cdot $ \\
$ \cdot $ & & $ \cdot $
\psset{shortput=nab,arrows=-}
\small
\ncline[linestyle=dotted,arrows=-,nodesep=15pt]{1,5}{1,7}
\ncline{1,2}{2,1}
\ncline{1,2}{2,3}
\endpsmatrix
~~~~
\psmatrix[mnode=circle,colsep=0.2,rowsep=0.2]
& $ \cdot $ \\
$ \cdot $ & & $ \cdot $
\psset{shortput=nab,arrows=-}
\small
\ncline[linestyle=dotted,arrows=-,nodesep=15pt]{1,5}{1,7}
\ncline{1,2}{2,3}
\ncline{2,1}{2,3}
\endpsmatrix
~~~~
\psmatrix[mnode=circle,colsep=0.2,rowsep=0.2]
& $ \cdot $ \\
$ \cdot $ & & $ \cdot $
\psset{shortput=nab,arrows=-}
\small
\ncline[linestyle=dotted,arrows=-,nodesep=15pt]{1,5}{1,7}
\ncline{1,2}{2,1}
\ncline{2,1}{2,3}
\endpsmatrix

\item $ n = 4 $: 8 paths

\psmatrix[mnode=circle,colsep=0.2,rowsep=0.2]
$ \cdot $ & $ \cdot $ \\
$ \cdot $ & $ \cdot $
\psset{shortput=nab,arrows=-}
\small
\ncline[linestyle=dotted,arrows=-,nodesep=15pt]{1,5}{1,7}
\ncline{1,1}{2,1}
\ncline{1,1}{1,2}
\ncline{2,1}{2,2}
\endpsmatrix
~~~~
\psmatrix[mnode=circle,colsep=0.2,rowsep=0.2]
$ \cdot $ & $ \cdot $ \\
$ \cdot $ & $ \cdot $
\psset{shortput=nab,arrows=-}
\small
\ncline[linestyle=dotted,arrows=-,nodesep=15pt]{1,5}{1,7}
\ncline{1,1}{2,1}
\ncline{2,1}{2,2}
\ncline{1,2}{2,2}
\endpsmatrix
~~~~
\psmatrix[mnode=circle,colsep=0.2,rowsep=0.2]
$ \cdot $ & $ \cdot $ \\
$ \cdot $ & $ \cdot $
\psset{shortput=nab,arrows=-}
\small
\ncline[linestyle=dotted,arrows=-,nodesep=15pt]{1,5}{1,7}
\ncline{1,1}{1,2}
\ncline{1,2}{2,2}
\ncline{2,1}{2,2}
\endpsmatrix
~~~~
\psmatrix[mnode=circle,colsep=0.2,rowsep=0.2]
$ \cdot $ & $ \cdot $ \\
$ \cdot $ & $ \cdot $
\psset{shortput=nab,arrows=-}
\small
\ncline[linestyle=dotted,arrows=-,nodesep=15pt]{1,5}{1,7}
\ncline{1,1}{1,2}
\ncline{1,1}{2,1}
\ncline{2,1}{2,2}
\endpsmatrix

\psmatrix[mnode=circle,colsep=0.2,rowsep=0.2]
$ \cdot $ & $ \cdot $ \\
$ \cdot $ & $ \cdot $
\psset{shortput=nab,arrows=-}
\small
\ncline[linestyle=dotted,arrows=-,nodesep=15pt]{1,5}{1,7}
\ncline{1,1}{1,2}
\ncline{1,1}{2,2}
\ncline{2,1}{2,2}
\endpsmatrix
~~~~
\psmatrix[mnode=circle,colsep=0.2,rowsep=0.2]
$ \cdot $ & $ \cdot $ \\
$ \cdot $ & $ \cdot $
\psset{shortput=nab,arrows=-}
\small
\ncline[linestyle=dotted,arrows=-,nodesep=15pt]{1,5}{1,7}
\ncline{1,1}{1,2}
\ncline{1,2}{2,1}
\ncline{2,1}{2,2}
\endpsmatrix
~~~~
\psmatrix[mnode=circle,colsep=0.2,rowsep=0.2]
$ \cdot $ & $ \cdot $ \\
$ \cdot $ & $ \cdot $
\psset{shortput=nab,arrows=-}
\small
\ncline[linestyle=dotted,arrows=-,nodesep=15pt]{1,5}{1,7}
\ncline{1,1}{2,1}
\ncline{1,1}{2,2}
\ncline{1,2}{2,2}
\endpsmatrix
~~~~
\psmatrix[mnode=circle,colsep=0.2,rowsep=0.2]
$ \cdot $ & $ \cdot $ \\
$ \cdot $ & $ \cdot $
\psset{shortput=nab,arrows=-}
\small
\ncline[linestyle=dotted,arrows=-,nodesep=15pt]{1,5}{1,7}
\ncline{1,1}{2,1}
\ncline{1,2}{2,1}
\ncline{1,2}{2,2}
\endpsmatrix

%%% n = 5
\item $ n = 5 $:
	\begin{itemize}
	\item more possibilites $ \rightarrow $ we go towards counting rather than enumerating
	\item look at possible types of paths, in CW and CCW direction for each of the five starting points. Divide by 2 for identical paths obtained with different starting points (since paths have 2 ends)
	\end{itemize}

\end{itemize}

\subsection{Example 2: Blokus}

\paragraph{Definition}
\begin{itemize}
\item consists of unit squares connected via edges (not just connected through vertices)
\item $ n $-polyomino has $ n $ unit squares
\item Animal: does not contain holes
\item free: rotation/mirroring is not taken into account
\item fixed: rotated/mirrored polyominoes are regarded as being distinct
\end{itemize}

\paragraph{Question:} How many $ n $-polyominoes are there?

\begin{itemize}
\item $ n = 1 $:

	$ \square $
	
	1
\item $ n = 2 $:
	\begin{itemize}
	\item $ \square \square $
	\item $ \begin{array}{c}\square \\ \square \end{array} $
	\end{itemize}
	
	fixed: 2, free: 1
\item $ n = 3 $:
	\begin{itemize}
	\item $ \begin{array}{c c c}\square & \square & \square \end{array} $
	\item $ \begin{array}{c}\square \\ \square \\ \square \end{array} $
	\item $ \begin{array}{c c}\square & \square \\ \square \end{array} ~~ \times 4 $
	\end{itemize}
	
	fixed: 6, free: 2

\item $ n = 4 $:
	\begin{itemize}
	\item $ \begin{array}{c c c c}\square & \square & \square & \square \end{array} ~~ \times 2 $
	\item $ \begin{array}{c c}\square \\ \square \\ \square & \square \end{array} ~~ \times 8 $
	\item $ \begin{array}{c c c}\square & \square & \square \\ & \square \end{array} ~~ \times 4 $ (reflection has same effect as rotation)
	\item $ \begin{array}{c c} & \square \\ \square & \square \\ \square \end{array} ~~ \times 4 $
	\item $ \begin{array}{c c}\square & \square \\ \square & \square \end{array} ~~ \times 1 $
	\end{itemize}
	
	fixed: 19, free: 5

\end{itemize}


Enumerate/Construct all polyominoes
\begin{itemize}
\item construct the $ (n + 1) $-ominoes from the $ n $-ominoes. This works because we can show that from each $ (n + 1) $-ominoes you can remove a square and get a valid $ n $-omino (idea: construct an arbitrary spanning tree, remove leaf)
\item find and remove duplicates
\item how many $ (n + 1) $-ominoes are generated?
\end{itemize}

	maximum number:
	$ 2n + 2 $ or $ 4n - 2 \dot | E | $
	
	Why?
	
	$ \square $: 4 possibilities to add a square (at each edge)
	
	$\begin{array}{c c c c}
	& & \square \\
	\square & \square & \square & \square \\
	& \square
	\end{array} $
	
	$ 4n - 2 \dot | E | \rightarrow 4n - 2(n - 1) = 2(n + 2) $
	
	How to find duplicates? compare every pair $ \rightarrow O(k^2 n) $

\subsubsection{Fingerprints}

\begin{itemize}

\item maps an object (here: polyomino) to a \textit{unique} representation, independent of translation, rotation, reflection

\item sort the fingerprints (insertion sort)

\item binary search

\end{itemize}

One possibility for polyominoes: use vector representations on a coordinate system \newline

Examples:
\begin{itemize}
\item $
\begin{array}{l c c c}
3 & & & \square \\
2 & & & \square \\
1 & \square & \square & \square \\
& 1 & 2 & 3 \\
\end{array}
$
has vector representation $ <(1, 1), (2, 1), (3, 1), (3, 2), (3, 3)> $

$
\begin{array}{l c c c}
3 & \square \\
2 & \square \\
1 & \square & \square & \square \\
& 1 & 2 & 3 \\
\end{array}
$
has vector representation
$  <(1, 1), (2, 1), (3, 1), (1, 2), (1, 3)> $

This is lexicographically smaller than the first one

\item 
$
\begin{array}{l c c c}
3 & \square \\
2 & \square & \square \\
1 & & \square & \square \\
& 1 & 2 & 3 \\
\end{array}
~~~~
\begin{array}{l c c c}
3 & & \square & \square \\
2 & \square & \square \\
1 & \square \\
& 1 & 2 & 3 \\
\end{array}
$
$ <(1, 1), (1, 2), (2, 2), (2, 3), (3, 3)> $ \bigskip

\item
$
\begin{array}{l c c c}
3 \\
2 & \square & \square & \square \\
1 & \square & & \square \\
& 1 & 2 & 3 \\
\end{array}
~~~~
\begin{array}{l c c c}
3 & \square & \square \\
2 & \square \\
1 & \square & \square \\
& 1 & 2 & 3 \\
\end{array}
$
$ <(1, 1), (3, 1), (1, 2), (1, 3), (2, 3)> $
\end{itemize}

Generally:
\begin{enumerate}
\item generate all 8 transformations (4 rotations, 2 reflections)
\item take the lexicographically smallest one $ \rightarrow O(1) \cdot O(n) $ for an $ n $-omino
\end{enumerate}

$ f'(n) $ fingerprints generated in total $ \rightarrow f'(n)\left( O(n) + \log (f'(n) \cdot O(n) \right)$ \newline

$ \lim \limits_{n \rightarrow \infty} \frac{f ( n + 1)}{f(n)} = \lambda $, $ 3.98 < \lambda < 4.65 $ \newline

$ f(n) $ is known up to $ n = 56 $.



%%%%%% 17. März

\subsection{Example Spanning Trees on Ladders}

with $ n $ rangs

%%% figure

\paragraph{Question:} Number of spanning trees on a ladder with $ n $ rangs

\begin{itemize}

\item $ n = 1 $

\psmatrix[mnode=circle,colsep=0.2,rowsep=0.2]
$ \cdot $ & $ \cdot $ & $ \cdot $ \\
$ \cdot $ & $ \cdot $ & $ \cdot $
\psset{shortput=nab,arrows=-}
\ncline{1,1}{1,2}
\ncline{1,2}{1,3}
\ncline{2,1}{2,2}
\ncline{2,2}{2,3}
\ncline{1,2}{2,2}
\endpsmatrix \smallskip

\# = 1

\item $ n = 2 $

\psmatrix[mnode=circle,colsep=0.2,rowsep=0.2]
$ \cdot $ & $ \cdot $ & $ \cdot $ & $ \cdot $ \\
$ \cdot $ & $ \cdot $ & $ \cdot $ & $ \cdot $
\psset{shortput=nab,arrows=-}
\ncline{1,1}{1,2}
\ncline[linestyle=dotted]{1,2}{1,3}
\ncline{1,3}{1,4}
\ncline{2,1}{2,2}
\ncline[linestyle=dotted]{2,2}{2,3}
\ncline{2,3}{2,4}
\ncline[linestyle=dotted]{1,2}{2,2}
\ncline[linestyle=dotted]{1,3}{2,3}
\endpsmatrix \smallskip

\# = 4

\item $ n = 3 $

\psmatrix[mnode=circle,colsep=0.2,rowsep=0.2]
$ \cdot $ & $ \cdot $ & $ \cdot $ & $ \cdot $ & $ \cdot $ \\
$ \cdot $ & $ \cdot $ & $ \cdot $ & $ \cdot $ & $ \cdot $
\psset{shortput=nab,arrows=-}
\ncline{1,1}{1,2}
\ncline{1,2}{1,3}
\ncline{1,3}{1,4}
\ncline{2,1}{2,2}
\ncline{2,2}{2,3}
\ncline{2,3}{2,4}
\ncline{1,2}{2,2}
\ncline{1,3}{2,3}
\ncline{1,4}{2,4}
\ncline{1,4}{1,5}
\ncline{2,4}{2,5}
\endpsmatrix
\hspace{20pt} $ \Leftrightarrow $ \hspace{20pt}
\psmatrix[mnode=circle,colsep=0.2,rowsep=0.2]
$ \cdot $ & $ \cdot $ & $ \cdot $ & $ \cdot $ \\
$ \cdot $ & $ \cdot $ & $ \cdot $ & $ \cdot $
\psset{shortput=nab,arrows=-}
\ncline{1,1}{1,2}
\ncline{1,1}{2,1}
\ncline{1,2}{1,3}
\ncline{1,3}{1,4}
\ncline{2,1}{2,2}
\ncline{2,2}{2,3}
\ncline{2,3}{2,4}
\ncline{1,2}{2,2}
\ncline{1,3}{2,3}
\ncline{1,4}{2,4}
\endpsmatrix \smallskip

\# = 15

\end{itemize}

\paragraph{Answer:}
Counting spanning trees on a ladder (with $ n $ rangs) $ \Leftrightarrow $ counting spanning trees on a $ 2n $ grid

$ n \rightarrow n + 1 $

%%% figure

\begin{enumerate}

\item
	The 2 rightmost vertices for $ n $ share a connected component $ \rightarrow A(n) $
	
\item
	The 2 rightmost vertices for $ n $ are in different connected components $ \rightarrow B(n) $

\end{enumerate}

Consider case 1
%% figure

$ A(n + 1) = 3 A(n) + 1 B(n) $

Consider case 2

%% figure

% $ B(n + 1) = 2 A(n - 1) + B(n - 1) 

%% .....

\[ A(1) = \frac{1}{2 \sqrt{3}} \left( ( 2 + \sqrt{3} )^1 - ( 2 - \sqrt{3} )^1 \right) = \frac{2 \sqrt{3}}{2 \sqrt{3}} \]
\[ A(2) = \dots = \frac{8 \sqrt{3}}{2 \sqrt{3}} = 4 \] \medskip

\begin{eqnarray}
A(n) = 4 A(n - 1) - A(n - 2) \\
= \frac{4}{2 \sqrt{3}} \left( ( 2 + \sqrt{3} )^{n-1} - ( 2 - \sqrt{3} )^{n-1} \right) - \frac{1}{2 \sqrt{3}} \left( ( 2 + \sqrt{3} )^{n-2} - ( 2 - \sqrt{3} )^{n-2} \right)
\end{eqnarray}

%%%%....

\section{Inclusion – Exclusion Principle}

\subsection{Example: Multipliers}

$ S = \{1, \dots, 100\} $

\paragraph{Question:} How many numbers in $ S $ are multipliers of 6 or 7? \newline

$ | S_6 | = \lfloor \frac{100}{6} \rfloor $ = 16, $ | S_6 | = 14 $ \newline

$ | S_{6,7} | \neq | S_6 | + | S_6 | $: \newline %% figure

\begin{pspicture}(6,2)
\rput[c](0.6,1.5){\rnode{1}{$S_6$}}
\rput[c](4.9,1.5){\rnode{2}{$S_7$}}
\pscircle(2,1){1}
\pscircle(3.5,1){1}
\end{pspicture}


$ | S_{6,7} | = | S_6 | + | S_6 | - | S_{6 \& 7} | $

\subsection{Example: Relatively Prime Numbers}

$ S = \{1, \dots, 180\} $

\paragraph{Question:} How many numbers in $ S $ are relatively prime to 180?

\paragraph{Idea:} Count all numbers \textit{not} relatively prime to 180 $ \rightarrow T $.

$ | T | = ? $

\begin{pspicture}(6,4)
\rput[c](0.6,3.5){\rnode{1}{$T_2$}}
\rput[c](4.9,3.5){\rnode{2}{$T_5$}}
\rput[c](2.75,0){\rnode{2}{$T_3$}}
\pscircle(2,2.5){1}
\pscircle(3.5,2.5){1}
\pscircle(2.75,1.5){1}
\end{pspicture}

$ | T_2 | = 90 $

$ | T_2 | = 60 $

$ | T_2 | = 36 $ \newline

$ | T | = | T_2 | + | T_3 | + | T_5 | - (| T_2 \cap T_3 | + | T_2 \cap T_5 | + | T_3 \cap T_5 |) + | T_2 \cap T_3  \cap T_5 | $

$ | T | = 90 + 60 + 36 - (30 + 18 + 12) + 6 = 132 $ \newline

$ | S | - | T | = 48 $ \newline

In general: \newline

\[ \mathcal{A} = \{ A_i \}^n_{i=1}, I = \{1, \dots, n\} \]
\[ |_{i \in I} A_i | = \sum \limits_{i=1}^n \left( ( -1)^{i - 1} \sum \limits_{J \subseteq I, |J| = i} | \bigcap \limits_{j \in J} A_j | \right) \]

\paragraph{Proof:}

$ x \in A _1 \cup \dots \cup A_n $, $ x $ is contained in $ m $ sets $ A_i $.
\[ | A_1 \cup \dots \cup A_n | = \sum (.) - \sum (. \cap .) + \sum (. \cap . \cap .) - \sum (. \cap . \cap . \cap .) \dots \]

\begin{enumerate}
\item for (single) sets \textit{not} containing $ x \rightarrow + 0 $
\item if in any $ A_i \cap \dots \cap A_j $ at least one of the sets does \textit{not} contain $ x \rightarrow \pm 0 $
\item $ \rightarrow $ only consider intersections of sets where all sets contain $ x $

	$ \begin{pmatrix}m \\ k \end{pmatrix} $ intersections of $ k $ sets, all sets containing $ x \rightarrow \pm 1 $
	
	\[ \Rightarrow + \begin{pmatrix}m \\ 1 \end{pmatrix} -  \begin{pmatrix}m \\ 2 \end{pmatrix} + \dots + (-1)^{m - 1}  \begin{pmatrix}m \\ m \end{pmatrix} = 1 \]
	\[ \begin{pmatrix}m \\ 0 \end{pmatrix} + \begin{pmatrix}m \\ 1 \end{pmatrix} -  \begin{pmatrix}m \\ 2 \end{pmatrix} + \dots + (-1)^{m}  \begin{pmatrix}m \\ m \end{pmatrix} = 0 \]
	
	Pascal's triangle
	
	\psmatrix[colsep=0.4cm,rowsep=0.4cm]
	& 1 & & 3 & & 3 & & 1 \\
	1 & & 4 & & 6 & & 4 & & 1
	\psset{arrows=->,nodesep=2pt}
	\ncline{1,2}{2,1}
	\ncline{1,2}{2,3}
	\ncline{1,4}{2,3}
	\ncline{1,4}{2,5}
	\ncline{1,6}{2,5}
	\ncline{1,6}{2,7}
	\ncline{1,8}{2,7}
	\ncline{1,8}{2,9}
	\endpsmatrix
\end{enumerate}

\subsection{Example: Number of Words}

\paragraph{Given:} Language $ L $ with alphabet $ \Sigma = \{ A, B, C \} $ and every word of $ L $ consists of 10 characters.

\paragraph{Question:} How many words of $ L $ contain each character at least once?

\[ | G | = | L | - | X | = 3^{10} - (3 \cdot 2^{10} - 3 + 0) = 55980 \]

\[ | S_A | = | S_B | = | S_C | = 2^{10} \]
\[ | S_A \cap S_B | = | S_A \cap S_C | = | S_B \cap S_C | = 1 \]
\[ | S_A \cap S_B \cap S_C | = 0 \]
\[ | X | = | S_A | + | S_B | + | S_C | - | S_A \cap S_B |  - | S_A \cap S_C | - | S_B \cap S_C | + | S_A \cap S_B \cap S_C | \]

\section{The Pigeon Hole Principle}

If $ n $ elements are distributed to $ k $ sets, then there exists at least one set containing at least $ \lceil \frac{n}{k} \rceil $ elements.

\subsection{Example: People in Vienna}

\paragraph{Claim:} In Vienna there exist at least two persons with the exact same number of hairs on their head.

\begin{itemize}
\item human: at most $ \approx 500,000 $ hairs
\item Vienna: $ \geq 1.7 \cdot 10^6 $ citizens
\end{itemize}

\paragraph{Idea:} Take 500,000 people. If 2 of them have the same amount of hair, then we are done. If not, add one more person, whose amount of hair must then already be present.

\subsection{Example: Division of odd numbers}

\paragraph{Given:} $ 1, 3, 7, 15, 31, \dots; a_i := 2^i - 1, i \geq 1 $ and $ q > 0 $ arbitrary odd number.

\paragraph{Claim:} At least one of the $ a_i $'s is (integer) divisible by $ q $.

\paragraph{Proof:} Consider all $ a_i $, $ 1 \leq i \leq q $
\begin{itemize}
\item If one of these $ a_i $ is divisible by $ q $, then we are done
\item otherwise: $ a_i = d_i \cdot q + r_i $, $ 0 < r_i < q $; $ q \geq n > m > 0 $

	$ \Rightarrow (q - 1) $ remainders $ \overset{\Rightarrow}{\text{p.h.p.}} \exists r_m, r_n $ such that $ r_m = r_n $
	
	Then $ (a_n - a_m) = (d_n - d_m) q + \underbrace{(r_n - r_m)}_{0} $
	
	$ (2^n - 1) - (2^m - 1) = 2^n - 2^m = 2^m(\underbrace{2^{n - m} - 1}_{= a_{n - m}}) $.
	
	$ 2^m $ is even and thus not divisible by $ q $.
\end{itemize}

\paragraph{In general:} if $ A_1, \dots A_k $ are finite (pairwise disjoint) sets \textit{and} $ | A_1 \cup \dots \cup A_k | > kr $, then $ \exists i $ such that $ | A_i | > r $.

\paragraph{Proof:} Assume for the sake of contradiction that all $ | A_i | \leq r $

$ k \cdot r \geq \sum \limits_{i = 1}^k | A_i | \geq | A_1 \cup \dots \cup A_k | > k \cdot r $

\subsection{Example: Lossless Data Compression}

\dots cannot guarantee compression for all input data!

\begin{itemize}
\item file $ \rightarrow $ string of bits
\item assume that each file is transformed into a \textit{distinct} file that is not larger
\item let $ F $ be the file with the least number of bits ($M$) that compresses $\rightarrow N $ bits
\item $ N < M : 2^N + 1 $ – cannot be all distinct
\end{itemize}


\section{Reverse Search}

\begin{itemize}
\item Avis, Fukuda 1992
\item enumerating data/objects/elements with almost no structure
\item algorithmic
\item count all objects \textit{exactly} once, no overcounting
\end{itemize}

\subsection{Example: Triangulations}

\paragraph{Given:} set $ S $ of labelled points in convex position

\paragraph{Question:} \# of triangulations on $ S $.

\begin{itemize}
\item $ n = 4 $

\psmatrix[emnode=circle,mnode=circle,colsep=0.5,rowsep=0.3,fillstyle=solid,fillcolor=black]
 &  \\
 &
\psset{shortput=nab,arrows=-}
\ncline{1,1}{1,2}
\ncline{1,1}{2,1}
\ncline{1,2}{2,2}
\ncline{2,1}{2,2}
\ncline{1,2}{2,1}
\endpsmatrix \hspace{10pt}
\psmatrix[emnode=circle,mnode=circle,colsep=0.5,rowsep=0.3,fillstyle=solid,fillcolor=black]
 &  \\
 &
\psset{shortput=nab,arrows=-}
\ncline{1,1}{1,2}
\ncline{1,1}{2,1}
\ncline{1,2}{2,2}
\ncline{2,1}{2,2}
\ncline{1,1}{2,2}
\endpsmatrix \smallskip

\item $ n = 5 $

\begin{pspicture}(2,2)
\rput[c](1,1){\PstPentagon[PolyName = A]}
\psline(A1)(A3)
\psline(A3)(A5)
\end{pspicture}
$ \rightarrow $
\begin{pspicture}(2,2)
\rput[c](1,1){\PstPentagon[PolyName = A]}
\psline(A1)(A3)
\psline(A4)(A1)
\end{pspicture}
$ \rightarrow $
\begin{pspicture}(2,2)
\rput[c](1,1){\PstPentagon[PolyName = A]}
\psline(A2)(A4)
\psline(A4)(A1)
\end{pspicture}
$ \rightarrow $
\begin{pspicture}(2,2)
\rput[c](1,1){\PstPentagon[PolyName = A]}
\psline(A2)(A4)
\psline(A2)(A5)
\end{pspicture}
$ \rightarrow $
\begin{pspicture}(2,2)
\rput[c](1,1){\PstPentagon[PolyName = A]}
\psline(A2)(A5)
\psline(A5)(A3)
\end{pspicture}
$ \curvearrowright $


\end{itemize}

Flip on triangulations:
4 points, remove the diagonal and add the other diagonal

\subsection{Basic Idea}

\begin{itemize}
\item abstract graph $ G(V, E) $
\item vertices $ V $: elements to be enumerated, $ | V | $ is very large
\item edges $ e \in E: e = v_1, v_2; v_1, v_2 \in V $
\item $ \Rightarrow G $ is an undirected graph
\item $ G $ must be connected
\item task: enumerate all vertices of $ G $
\item idea: breadth-first or depth-first search: $ O(|V| + |E|) $ which is ok, but too much space necessary
\end{itemize}


\subsection{Example: Poker Chips Stacking}

\paragraph{Given:} four colours (red, green, blue, white) of poker chips

\paragraph{Task:} build stacks of height $ \leq 30 $ such that at most 3 chips of equal colour appear directly after another

\begin{itemize}
\item $ v_\Phi $: empty stack
\item neighbours:
	\begin{itemize}
	\item 'successors': add a chip on top of the stack
	\item 'predecessors': remove the topmost chip from the stack
	\end{itemize}
\item depth-first search
	
	\pstree[arrows=-]{\Toval{\textit{stack}}}{
		\pstree{\Toval{red}}{
			\Toval{$ \checkmark $}
			\Toval{$ \checkmark $}
			\Toval{$ \checkmark $}
			\Toval{$ \checkmark $}
		}
		\pstree{\Toval{green}}{
			\Toval{$ \checkmark $}
			\Toval{$ \checkmark $}
			\Toval{$ \checkmark $}
			\Toval{$ \checkmark $}
		}
		\Toval{blue}
		\Toval{white}
	}
\item store current stack.

	Nodes that have already been visited can be deduced from the stack composition.
\end{itemize}

\paragraph{Requirements:}

\begin{itemize}
\item unique root
\item successors and predecessors in arbitrary but fixed order

	neighbour relation $ \gamma(v, k) \in \Gamma(v) $ ($ k $th neighbour)
\item a unique predecessor function $ f(v), f: v \rightarrow v $ in $ G(V, E) $
	\begin{enumerate}
	\item $ \exists v_0 \in V: f(v_0) = v_0 $: root
	\item $ \forall v \in V \backslash \{v_0\}: f(v) \in \Gamma(v) $: exactly one root
	\item $ \forall v \in V \backslash \{v_0\} \forall x \in \mathbb{N}: f^x(v) \neq v $: cycle-free ($ f^x = f(f \dots f(v) \dots) $) $ x $ times
	
		$ \rightarrow $ reaching the root in finite time $ f^{|V|}(v) = v_0 $
	\end{enumerate}

\end{itemize}

\subsection{Example: Tic Tac Toe}

\paragraph{Given:}

\paragraph{Task:} Enumerate all valid game positions (not considering symmetries)

\begin{itemize}
\item $ V $: game positions
\item labelled board cells
\item neighbours $ \gamma (v, k) $: legally add $ k $-th mark or remove one
\item predecessor function: empty board as root, otherwise remove highest legal mark
\item when iterating over the possible positions, check each node's predecessor to see if it has been visited before
\end{itemize}


\subsection{Example: Connect Four (4 Gewinnt)}

\begin{tabular}{| c | c | c | c | c | c |}
\vdots & \vdots & \vdots & \vdots & \vdots & \vdots \\
\hline
& & & & & \\
\hline
& & & X & & \\
\hline
& & & O & & \\
\hline
\end{tabular} \newline

too complex for reverse search, NP-hard

\subsection{Algorithm}

%%% part 1

%%% part 2

Complexity Analysis

\subsection{Example: Triangulations}

\end{document}